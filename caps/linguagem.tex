\chapter{Linguagem}
\indent \textit{Python} é um linguagem de programação que se tornou muito popular por possuir muitas vantagens, com destaque para sua sintaxe. \cite{menezes2010introduccao} destaca o \textit{Python} como uma linguagem interessante para se iniciar a programar, devido sua clareza, simplicidade e que pode ser usada para construir grandes projetos. Como característica \textit{Python} pode ser descrita como uma linguagem de programação de alto nível, já que possui uma sintaxe mais próxima da escrita humana, além de ser interpretada, além de ter suporte para os paradigmas orientado a objetos, funcional, imperativo e procedural, com uma tipagem dinâmica e forte e de propósito geral \cite{wiki:python}.

O \textit{Python} enfatiza a produtividade e simplicidade, o que permite aos programadores expressarem conceitos com menos código, quando se compara com linguagens como java ou C++. A linguagem também contém uma vasta biblioteca padrão e ecossistema de pacotes de terceiros o que contribui muito para a sua capacidade de aplicações.\cite{wiki:python}

O projeto do \textit{Python} é open source, mantido por uma comunidade global ativa, com o código-fonte disponível para modificações e distribuição sob a licença PSF (\textit{Python Software Foundation}). Além disto, o \textit{Python} também é multiplataforma, abrangendo os sistemas operacionais Windows, macOS, Linux e várias versões do Unix.\cite{wiki:python}

\section{História}

\indent \textit{Python} foi criada no final dos anos 80 por Guido van Rossum no Centrum Wiskunde \& Informatica (CWI) na Holanda, como uma sucessora da Linguagem de programação ABC, que era simples, mas limitada em funcionalidades, que por sua vez era inspirada por SETL. O \textit{Python} foi projetado para superar as limitações das linguagens anteriores, incorporando recursos que incentivassem a produtividade e a simplicidade. Seu lançamento veio a ocorrer pela primeira vez em 1991, com a versão 0.9.0, que já incluía muitos dos principais recursos que caracterizam a linguagem até os dias atuais. O \textit{Python} 1.0, lançado em janeiro de 1994, marcou a primeira versão oficial e estável da linguagem.\cite{wiki:python}  Nós dias atuais a linguagem é constantemente classificada como uma das linguagens de programação mais utilizadas e com mais popularidade, com destaque para a comunidade de aprendizado de máquina. \cite{wiki:python}